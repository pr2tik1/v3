% Options for packages loaded elsewhere
\PassOptionsToPackage{unicode}{hyperref}
\PassOptionsToPackage{hyphens}{url}
\PassOptionsToPackage{dvipsnames,svgnames,x11names}{xcolor}
%
\documentclass[
  letterpaper,
  DIV=11,
  numbers=noendperiod]{scrartcl}

\usepackage{amsmath,amssymb}
\usepackage{iftex}
\ifPDFTeX
  \usepackage[T1]{fontenc}
  \usepackage[utf8]{inputenc}
  \usepackage{textcomp} % provide euro and other symbols
\else % if luatex or xetex
  \usepackage{unicode-math}
  \defaultfontfeatures{Scale=MatchLowercase}
  \defaultfontfeatures[\rmfamily]{Ligatures=TeX,Scale=1}
\fi
\usepackage{lmodern}
\ifPDFTeX\else  
    % xetex/luatex font selection
\fi
% Use upquote if available, for straight quotes in verbatim environments
\IfFileExists{upquote.sty}{\usepackage{upquote}}{}
\IfFileExists{microtype.sty}{% use microtype if available
  \usepackage[]{microtype}
  \UseMicrotypeSet[protrusion]{basicmath} % disable protrusion for tt fonts
}{}
\makeatletter
\@ifundefined{KOMAClassName}{% if non-KOMA class
  \IfFileExists{parskip.sty}{%
    \usepackage{parskip}
  }{% else
    \setlength{\parindent}{0pt}
    \setlength{\parskip}{6pt plus 2pt minus 1pt}}
}{% if KOMA class
  \KOMAoptions{parskip=half}}
\makeatother
\usepackage{xcolor}
\setlength{\emergencystretch}{3em} % prevent overfull lines
\setcounter{secnumdepth}{-\maxdimen} % remove section numbering
% Make \paragraph and \subparagraph free-standing
\ifx\paragraph\undefined\else
  \let\oldparagraph\paragraph
  \renewcommand{\paragraph}[1]{\oldparagraph{#1}\mbox{}}
\fi
\ifx\subparagraph\undefined\else
  \let\oldsubparagraph\subparagraph
  \renewcommand{\subparagraph}[1]{\oldsubparagraph{#1}\mbox{}}
\fi

\usepackage{color}
\usepackage{fancyvrb}
\newcommand{\VerbBar}{|}
\newcommand{\VERB}{\Verb[commandchars=\\\{\}]}
\DefineVerbatimEnvironment{Highlighting}{Verbatim}{commandchars=\\\{\}}
% Add ',fontsize=\small' for more characters per line
\usepackage{framed}
\definecolor{shadecolor}{RGB}{241,243,245}
\newenvironment{Shaded}{\begin{snugshade}}{\end{snugshade}}
\newcommand{\AlertTok}[1]{\textcolor[rgb]{0.68,0.00,0.00}{#1}}
\newcommand{\AnnotationTok}[1]{\textcolor[rgb]{0.37,0.37,0.37}{#1}}
\newcommand{\AttributeTok}[1]{\textcolor[rgb]{0.40,0.45,0.13}{#1}}
\newcommand{\BaseNTok}[1]{\textcolor[rgb]{0.68,0.00,0.00}{#1}}
\newcommand{\BuiltInTok}[1]{\textcolor[rgb]{0.00,0.23,0.31}{#1}}
\newcommand{\CharTok}[1]{\textcolor[rgb]{0.13,0.47,0.30}{#1}}
\newcommand{\CommentTok}[1]{\textcolor[rgb]{0.37,0.37,0.37}{#1}}
\newcommand{\CommentVarTok}[1]{\textcolor[rgb]{0.37,0.37,0.37}{\textit{#1}}}
\newcommand{\ConstantTok}[1]{\textcolor[rgb]{0.56,0.35,0.01}{#1}}
\newcommand{\ControlFlowTok}[1]{\textcolor[rgb]{0.00,0.23,0.31}{#1}}
\newcommand{\DataTypeTok}[1]{\textcolor[rgb]{0.68,0.00,0.00}{#1}}
\newcommand{\DecValTok}[1]{\textcolor[rgb]{0.68,0.00,0.00}{#1}}
\newcommand{\DocumentationTok}[1]{\textcolor[rgb]{0.37,0.37,0.37}{\textit{#1}}}
\newcommand{\ErrorTok}[1]{\textcolor[rgb]{0.68,0.00,0.00}{#1}}
\newcommand{\ExtensionTok}[1]{\textcolor[rgb]{0.00,0.23,0.31}{#1}}
\newcommand{\FloatTok}[1]{\textcolor[rgb]{0.68,0.00,0.00}{#1}}
\newcommand{\FunctionTok}[1]{\textcolor[rgb]{0.28,0.35,0.67}{#1}}
\newcommand{\ImportTok}[1]{\textcolor[rgb]{0.00,0.46,0.62}{#1}}
\newcommand{\InformationTok}[1]{\textcolor[rgb]{0.37,0.37,0.37}{#1}}
\newcommand{\KeywordTok}[1]{\textcolor[rgb]{0.00,0.23,0.31}{#1}}
\newcommand{\NormalTok}[1]{\textcolor[rgb]{0.00,0.23,0.31}{#1}}
\newcommand{\OperatorTok}[1]{\textcolor[rgb]{0.37,0.37,0.37}{#1}}
\newcommand{\OtherTok}[1]{\textcolor[rgb]{0.00,0.23,0.31}{#1}}
\newcommand{\PreprocessorTok}[1]{\textcolor[rgb]{0.68,0.00,0.00}{#1}}
\newcommand{\RegionMarkerTok}[1]{\textcolor[rgb]{0.00,0.23,0.31}{#1}}
\newcommand{\SpecialCharTok}[1]{\textcolor[rgb]{0.37,0.37,0.37}{#1}}
\newcommand{\SpecialStringTok}[1]{\textcolor[rgb]{0.13,0.47,0.30}{#1}}
\newcommand{\StringTok}[1]{\textcolor[rgb]{0.13,0.47,0.30}{#1}}
\newcommand{\VariableTok}[1]{\textcolor[rgb]{0.07,0.07,0.07}{#1}}
\newcommand{\VerbatimStringTok}[1]{\textcolor[rgb]{0.13,0.47,0.30}{#1}}
\newcommand{\WarningTok}[1]{\textcolor[rgb]{0.37,0.37,0.37}{\textit{#1}}}

\providecommand{\tightlist}{%
  \setlength{\itemsep}{0pt}\setlength{\parskip}{0pt}}\usepackage{longtable,booktabs,array}
\usepackage{calc} % for calculating minipage widths
% Correct order of tables after \paragraph or \subparagraph
\usepackage{etoolbox}
\makeatletter
\patchcmd\longtable{\par}{\if@noskipsec\mbox{}\fi\par}{}{}
\makeatother
% Allow footnotes in longtable head/foot
\IfFileExists{footnotehyper.sty}{\usepackage{footnotehyper}}{\usepackage{footnote}}
\makesavenoteenv{longtable}
\usepackage{graphicx}
\makeatletter
\def\maxwidth{\ifdim\Gin@nat@width>\linewidth\linewidth\else\Gin@nat@width\fi}
\def\maxheight{\ifdim\Gin@nat@height>\textheight\textheight\else\Gin@nat@height\fi}
\makeatother
% Scale images if necessary, so that they will not overflow the page
% margins by default, and it is still possible to overwrite the defaults
% using explicit options in \includegraphics[width, height, ...]{}
\setkeys{Gin}{width=\maxwidth,height=\maxheight,keepaspectratio}
% Set default figure placement to htbp
\makeatletter
\def\fps@figure{htbp}
\makeatother

\KOMAoption{captions}{tableheading}
\makeatletter
\makeatother
\makeatletter
\makeatother
\makeatletter
\@ifpackageloaded{caption}{}{\usepackage{caption}}
\AtBeginDocument{%
\ifdefined\contentsname
  \renewcommand*\contentsname{Table of contents}
\else
  \newcommand\contentsname{Table of contents}
\fi
\ifdefined\listfigurename
  \renewcommand*\listfigurename{List of Figures}
\else
  \newcommand\listfigurename{List of Figures}
\fi
\ifdefined\listtablename
  \renewcommand*\listtablename{List of Tables}
\else
  \newcommand\listtablename{List of Tables}
\fi
\ifdefined\figurename
  \renewcommand*\figurename{Figure}
\else
  \newcommand\figurename{Figure}
\fi
\ifdefined\tablename
  \renewcommand*\tablename{Table}
\else
  \newcommand\tablename{Table}
\fi
}
\@ifpackageloaded{float}{}{\usepackage{float}}
\floatstyle{ruled}
\@ifundefined{c@chapter}{\newfloat{codelisting}{h}{lop}}{\newfloat{codelisting}{h}{lop}[chapter]}
\floatname{codelisting}{Listing}
\newcommand*\listoflistings{\listof{codelisting}{List of Listings}}
\makeatother
\makeatletter
\@ifpackageloaded{caption}{}{\usepackage{caption}}
\@ifpackageloaded{subcaption}{}{\usepackage{subcaption}}
\makeatother
\makeatletter
\@ifpackageloaded{tcolorbox}{}{\usepackage[skins,breakable]{tcolorbox}}
\makeatother
\makeatletter
\@ifundefined{shadecolor}{\definecolor{shadecolor}{rgb}{.97, .97, .97}}
\makeatother
\makeatletter
\makeatother
\makeatletter
\makeatother
\ifLuaTeX
  \usepackage{selnolig}  % disable illegal ligatures
\fi
\IfFileExists{bookmark.sty}{\usepackage{bookmark}}{\usepackage{hyperref}}
\IfFileExists{xurl.sty}{\usepackage{xurl}}{} % add URL line breaks if available
\urlstyle{same} % disable monospaced font for URLs
\hypersetup{
  pdftitle={Sunburst Charts - Plotly},
  colorlinks=true,
  linkcolor={blue},
  filecolor={Maroon},
  citecolor={Blue},
  urlcolor={Blue},
  pdfcreator={LaTeX via pandoc}}

\title{Sunburst Charts - Plotly}
\author{}
\date{}

\begin{document}
\maketitle
\ifdefined\Shaded\renewenvironment{Shaded}{\begin{tcolorbox}[frame hidden, interior hidden, boxrule=0pt, breakable, borderline west={3pt}{0pt}{shadecolor}, enhanced, sharp corners]}{\end{tcolorbox}}\fi

\hypertarget{a-introduction}{%
\subsection{(A) Introduction ☕️}\label{a-introduction}}

\begin{center}\rule{0.5\linewidth}{0.5pt}\end{center}

Data visualization plays a vital role in various domains such as data
analytics, data science, data dashboarding, and exploratory/statistical
analysis. Within the Python and R ecosystems, there are several popular
visualization libraries commonly used. These include:

\begin{itemize}
\tightlist
\item
  \href{https://matplotlib.org}{Matplotlib}
\item
  \href{https://seaborn.pydata.org}{Seaborn}
\item
  \href{https://plotly.com}{Plotly}
\item
  \href{https://altair-viz.github.io}{Altair}
\item
  \href{https://bokeh.org}{Bokeh}
\end{itemize}

Among these, the widely used library is the Plotly Graphing Library,
which offers libraries in multiple languages, high-quality
scientific/non-scientific graphs, and easily shareable interactive
plots.

In this post, I will be discussing an intriguing plot called the
Sunburst Chart. Sunburst charts provide an interactive visualization of
layered information, allowing for an enhanced understanding of complex
data structures.

\hypertarget{b-sunburst-chart}{%
\subsection{\texorpdfstring{(B) \textbf{Sunburst
Chart}}{(B) Sunburst Chart}}\label{b-sunburst-chart}}

\begin{center}\rule{0.5\linewidth}{0.5pt}\end{center}

A sunburst chart is a powerful visualization tool used to represent
hierarchical datasets. In a hierarchical dataset, there exists a
parent-child relationship among the features or variables, resembling a
tree-like structure. To generate a sunburst plot using Plotly, you can
leverage the capabilities of either plotly.express or
plotly.graph\_objects libraries.

Let's consider an example dataframe (dummy data for demonstration
purposes) with a tree-like structure, where the columns or features
exhibit parent-child relationships with other columns.

General Dataset: This dataframe contains classes and values organized in
columns, as depicted in the sample data provided. Sunburst DataFrame:
This hierarchical dataframe defines the logical parent-child
relationships between columns and their corresponding values.

Now, let's delve into how this data would appear by visualizing it using
a sunburst chart.

\hypertarget{c-datasets}{%
\subsection{(C) Datasets}\label{c-datasets}}

\begin{center}\rule{0.5\linewidth}{0.5pt}\end{center}

The following dataset is a dummy data for demonstration. Usually, you
may come accross, this kind of data while working on a data
science/analytics projects.

\begin{Shaded}
\begin{Highlighting}[]
\CommentTok{\#Importing pandas to handle dataframe}
\ImportTok{import}\NormalTok{ pandas }\ImportTok{as}\NormalTok{ pd}
\CommentTok{\# Suppress pandas warnings}
\ImportTok{import}\NormalTok{ warnings}
\NormalTok{warnings.filterwarnings(}\StringTok{"ignore"}\NormalTok{)}

\NormalTok{data }\OperatorTok{=}\NormalTok{ pd.read\_csv(}\StringTok{"../data/dummy\_data.csv"}\NormalTok{)}
\NormalTok{data}
\end{Highlighting}
\end{Shaded}

\begin{tabular}{llllr}
\toprule
{} & Country & State & City &  Population \\
\midrule
0  &   India &  INMP &   A1 &         512 \\
1  &   India &  INCG &   B2 &       12201 \\
2  &   India &  INCG &   M1 &        9021 \\
3  &     USA &  USNY &   C2 &         812 \\
4  &     USA &  USNY &   N1 &         821 \\
5  &     USA &  USNY &   O2 &         128 \\
6  &   India &  INSD &   D1 &        9104 \\
7  &   India &  INGD &   E2 &         132 \\
8  &     USA &  USSF &   F1 &          82 \\
9  &   India &  INSA &   G2 &        5121 \\
10 &   India &  INAS &   H1 &        1232 \\
11 &     USA &  USHF &   I2 &        8841 \\
12 &   India &  INSR &   J1 &          11 \\
13 &   India &  INCQ &   K2 &        1236 \\
14 &     USA &  USSF &   L3 &        1200 \\
15 &     USA &  USXO &   R1 &          60 \\
16 &     USA &  USXO &   Q1 &        4120 \\
17 &   India &  INGD &   S1 &        6012 \\
18 &   India &  INSD &   P1 &          20 \\
\bottomrule
\end{tabular}

The dataset is not in hierachical form. The sunburst chart needs a
parent, child and value variable for generating the plot. Hence, we need
to convert the table into a `chart-acceptable' format. The following
function performs the job. The function is modified version of original
function defined at Plotly's documentation, to know more about this
please visit
\href{https://plotly.com/python/sunburst-charts/\#:~:text=Charred-,Sunburst\%20chart\%20with\%20a\%20continuous\%20colorscale,-The\%20example\%20below}{here}.

\begin{Shaded}
\begin{Highlighting}[]
\KeywordTok{def}\NormalTok{ build\_hierarchical\_dataframe(df, levels, value\_column, metric):}
    \CommentTok{"""}
\CommentTok{    Build a hierarchy of levels for Sunburst.}
\CommentTok{    {-} Levels are given starting from the bottom to the top of the hierarchy,}
\CommentTok{    i.e. the last level corresponds to the root.}
\CommentTok{    {-} Input : }
\CommentTok{        {-} df : pandas dataframe }
\CommentTok{        {-} levels : list of column names in the order, child to root.}
\CommentTok{        {-} value\_column : string value corresponding to value of column to display in chart.}
\CommentTok{        {-} metric : string value equal to "sum" or "count".}
\CommentTok{    {-} Output:}
\CommentTok{        {-} df\_all\_trees : pandas dataframe for sunburst with columns, [\textquotesingle{}id\textquotesingle{}, \textquotesingle{}parent\textquotesingle{}, \textquotesingle{}value\textquotesingle{}].  }
\CommentTok{    """}
\NormalTok{    df\_all\_trees }\OperatorTok{=}\NormalTok{ pd.DataFrame(columns}\OperatorTok{=}\NormalTok{[}\StringTok{\textquotesingle{}id\textquotesingle{}}\NormalTok{, }\StringTok{\textquotesingle{}parent\textquotesingle{}}\NormalTok{, }\StringTok{\textquotesingle{}value\textquotesingle{}}\NormalTok{])}
    
    \ControlFlowTok{for}\NormalTok{ i, level }\KeywordTok{in} \BuiltInTok{enumerate}\NormalTok{(levels):}
\NormalTok{        df\_tree }\OperatorTok{=}\NormalTok{ pd.DataFrame(columns}\OperatorTok{=}\NormalTok{[}\StringTok{\textquotesingle{}id\textquotesingle{}}\NormalTok{, }\StringTok{\textquotesingle{}parent\textquotesingle{}}\NormalTok{, }\StringTok{\textquotesingle{}value\textquotesingle{}}\NormalTok{])}
        \CommentTok{\#\# Groupby based upon metric chosen}
        \ControlFlowTok{if}\NormalTok{ metric}\OperatorTok{==}\StringTok{"count"}\NormalTok{:}
\NormalTok{            dfg }\OperatorTok{=}\NormalTok{ df.groupby(levels[i:]).count()}
        \ControlFlowTok{else}\NormalTok{:}
\NormalTok{            dfg }\OperatorTok{=}\NormalTok{ df.groupby(levels[i:]).}\BuiltInTok{sum}\NormalTok{()}
        
\NormalTok{        dfg }\OperatorTok{=}\NormalTok{ dfg.reset\_index()}
\NormalTok{        df\_tree[}\StringTok{\textquotesingle{}id\textquotesingle{}}\NormalTok{] }\OperatorTok{=}\NormalTok{ dfg[level].copy()}

        \CommentTok{\#\# Set parent of the levels }
        \ControlFlowTok{if}\NormalTok{ i }\OperatorTok{\textless{}} \BuiltInTok{len}\NormalTok{(levels) }\OperatorTok{{-}} \DecValTok{1}\NormalTok{:}
\NormalTok{            df\_tree[}\StringTok{\textquotesingle{}parent\textquotesingle{}}\NormalTok{] }\OperatorTok{=}\NormalTok{ dfg[levels[i}\OperatorTok{+}\DecValTok{1}\NormalTok{]].copy()}
        \ControlFlowTok{else}\NormalTok{:}
\NormalTok{            df\_tree[}\StringTok{\textquotesingle{}parent\textquotesingle{}}\NormalTok{] }\OperatorTok{=} \StringTok{\textquotesingle{}Total\textquotesingle{}}
        
\NormalTok{        df\_tree[}\StringTok{\textquotesingle{}value\textquotesingle{}}\NormalTok{] }\OperatorTok{=}\NormalTok{ dfg[value\_column]}
\NormalTok{        df\_all\_trees }\OperatorTok{=}\NormalTok{ pd.concat([df\_all\_trees, df\_tree], ignore\_index}\OperatorTok{=}\VariableTok{True}\NormalTok{)}
    
    \CommentTok{\#\# Value calculation for parent }
    \ControlFlowTok{if}\NormalTok{ metric}\OperatorTok{==}\StringTok{"count"}\NormalTok{:}
\NormalTok{        total }\OperatorTok{=}\NormalTok{ pd.Series(}\BuiltInTok{dict}\NormalTok{(}\BuiltInTok{id}\OperatorTok{=}\StringTok{\textquotesingle{}Total\textquotesingle{}}\NormalTok{, parent}\OperatorTok{=}\StringTok{\textquotesingle{}\textquotesingle{}}\NormalTok{, value}\OperatorTok{=}\NormalTok{df[value\_column].count()))}
    \ControlFlowTok{else}\NormalTok{:}
\NormalTok{        total }\OperatorTok{=}\NormalTok{ pd.Series(}\BuiltInTok{dict}\NormalTok{(}\BuiltInTok{id}\OperatorTok{=}\StringTok{\textquotesingle{}Total\textquotesingle{}}\NormalTok{, parent}\OperatorTok{=}\StringTok{\textquotesingle{}\textquotesingle{}}\NormalTok{, value}\OperatorTok{=}\NormalTok{df[value\_column].}\BuiltInTok{sum}\NormalTok{()))}
    
    \CommentTok{\#\# Add frames one below the other to form the final dataframe}
\NormalTok{    df\_all\_trees }\OperatorTok{=}\NormalTok{ pd.concat([df\_all\_trees, pd.DataFrame([total])], ignore\_index}\OperatorTok{=}\VariableTok{True}\NormalTok{)}
    \ControlFlowTok{return}\NormalTok{ df\_all\_trees}
\end{Highlighting}
\end{Shaded}

\begin{Shaded}
\begin{Highlighting}[]
\NormalTok{levels }\OperatorTok{=}\NormalTok{ [}\StringTok{\textquotesingle{}City\textquotesingle{}}\NormalTok{, }\StringTok{\textquotesingle{}State\textquotesingle{}}\NormalTok{, }\StringTok{\textquotesingle{}Country\textquotesingle{}}\NormalTok{] }
\NormalTok{value\_column }\OperatorTok{=} \StringTok{\textquotesingle{}Population\textquotesingle{}}
\NormalTok{metric }\OperatorTok{=} \StringTok{"sum"}
\end{Highlighting}
\end{Shaded}

\hypertarget{hierarchical-sum-dataframe}{%
\subsubsection{Hierarchical Sum
dataframe}\label{hierarchical-sum-dataframe}}

This dataframe represents total population accross Country, State and
City under study.

\begin{Shaded}
\begin{Highlighting}[]
\NormalTok{df\_sum}\OperatorTok{=}\NormalTok{build\_hierarchical\_dataframe(data, levels, value\_column, metric}\OperatorTok{=}\StringTok{"sum"}\NormalTok{)}
\NormalTok{df\_sum}
\end{Highlighting}
\end{Shaded}

\begin{tabular}{llll}
\toprule
{} &     id & parent &  value \\
\midrule
0  &     A1 &   INMP &    512 \\
1  &     B2 &   INCG &  12201 \\
2  &     C2 &   USNY &    812 \\
3  &     D1 &   INSD &   9104 \\
4  &     E2 &   INGD &    132 \\
5  &     F1 &   USSF &     82 \\
6  &     G2 &   INSA &   5121 \\
7  &     H1 &   INAS &   1232 \\
8  &     I2 &   USHF &   8841 \\
9  &     J1 &   INSR &     11 \\
10 &     K2 &   INCQ &   1236 \\
11 &     L3 &   USSF &   1200 \\
12 &     M1 &   INCG &   9021 \\
13 &     N1 &   USNY &    821 \\
14 &     O2 &   USNY &    128 \\
15 &     P1 &   INSD &     20 \\
16 &     Q1 &   USXO &   4120 \\
17 &     R1 &   USXO &     60 \\
18 &     S1 &   INGD &   6012 \\
19 &   INAS &  India &   1232 \\
20 &   INCG &  India &  21222 \\
21 &   INCQ &  India &   1236 \\
22 &   INGD &  India &   6144 \\
23 &   INMP &  India &    512 \\
24 &   INSA &  India &   5121 \\
25 &   INSD &  India &   9124 \\
26 &   INSR &  India &     11 \\
27 &   USHF &    USA &   8841 \\
28 &   USNY &    USA &   1761 \\
29 &   USSF &    USA &   1282 \\
30 &   USXO &    USA &   4180 \\
31 &  India &  Total &  44602 \\
32 &    USA &  Total &  16064 \\
33 &  Total &        &  60666 \\
\bottomrule
\end{tabular}

\hypertarget{hierarchical-count-dataframe}{%
\subsubsection{Hierarchical Count
dataframe}\label{hierarchical-count-dataframe}}

This dataframe represents number of sub-classes (like City) accross
Country and State under study.

\begin{Shaded}
\begin{Highlighting}[]
\NormalTok{df\_count}\OperatorTok{=}\NormalTok{build\_hierarchical\_dataframe(data, levels, value\_column, metric}\OperatorTok{=}\StringTok{"count"}\NormalTok{)}
\NormalTok{df\_count}
\end{Highlighting}
\end{Shaded}

\begin{tabular}{llll}
\toprule
{} &     id & parent & value \\
\midrule
0  &     A1 &   INMP &     1 \\
1  &     B2 &   INCG &     1 \\
2  &     C2 &   USNY &     1 \\
3  &     D1 &   INSD &     1 \\
4  &     E2 &   INGD &     1 \\
5  &     F1 &   USSF &     1 \\
6  &     G2 &   INSA &     1 \\
7  &     H1 &   INAS &     1 \\
8  &     I2 &   USHF &     1 \\
9  &     J1 &   INSR &     1 \\
10 &     K2 &   INCQ &     1 \\
11 &     L3 &   USSF &     1 \\
12 &     M1 &   INCG &     1 \\
13 &     N1 &   USNY &     1 \\
14 &     O2 &   USNY &     1 \\
15 &     P1 &   INSD &     1 \\
16 &     Q1 &   USXO &     1 \\
17 &     R1 &   USXO &     1 \\
18 &     S1 &   INGD &     1 \\
19 &   INAS &  India &     1 \\
20 &   INCG &  India &     2 \\
21 &   INCQ &  India &     1 \\
22 &   INGD &  India &     2 \\
23 &   INMP &  India &     1 \\
24 &   INSA &  India &     1 \\
25 &   INSD &  India &     2 \\
26 &   INSR &  India &     1 \\
27 &   USHF &    USA &     1 \\
28 &   USNY &    USA &     3 \\
29 &   USSF &    USA &     2 \\
30 &   USXO &    USA &     2 \\
31 &  India &  Total &    11 \\
32 &    USA &  Total &     8 \\
33 &  Total &        &    19 \\
\bottomrule
\end{tabular}

\hypertarget{d-visualizations}{%
\subsection{(D) Visualizations}\label{d-visualizations}}

\begin{center}\rule{0.5\linewidth}{0.5pt}\end{center}

Now we would see the two most common ways of plotting sunburst charts in
python. The user can choose any of the following modules,

\begin{enumerate}
\def\labelenumi{\arabic{enumi}.}
\tightlist
\item
  Plotly Express
\item
  Plotly Graph Objects
\end{enumerate}

Both of these modules generate same \emph{``figure object''}. Just the
difference comes in code syntax and in flexibility of modifying graph as
required. Plotly express is more of generating plot by calling function
from already defined set of parameters. One may be more comfortable in
tweaking the details while working with graph objects. However, the
beauty of plotly is that you are able do the same things in the figure
generated from plotly express as those are possible in that with graph
objects.

We will be using both of them, and generate the plots for the datasets
generated in the above section.

\begin{Shaded}
\begin{Highlighting}[]
\ImportTok{from}\NormalTok{ io }\ImportTok{import}\NormalTok{ StringIO}
\ImportTok{from}\NormalTok{ IPython.display }\ImportTok{import}\NormalTok{ display\_html, HTML}
\end{Highlighting}
\end{Shaded}

\hypertarget{d.1.-plotly-express}{%
\subsubsection{(D.1.) Plotly Express}\label{d.1.-plotly-express}}

\begin{Shaded}
\begin{Highlighting}[]
\ImportTok{import}\NormalTok{ plotly.express }\ImportTok{as}\NormalTok{ px }

\NormalTok{figure }\OperatorTok{=}\NormalTok{ px.sunburst(data, path}\OperatorTok{=}\NormalTok{[}\StringTok{\textquotesingle{}Country\textquotesingle{}}\NormalTok{, }\StringTok{\textquotesingle{}State\textquotesingle{}}\NormalTok{, }\StringTok{\textquotesingle{}City\textquotesingle{}}\NormalTok{], values}\OperatorTok{=}\StringTok{\textquotesingle{}Population\textquotesingle{}}\NormalTok{)}
\NormalTok{figure.update\_layout(margin}\OperatorTok{=}\BuiltInTok{dict}\NormalTok{(t}\OperatorTok{=}\DecValTok{10}\NormalTok{, b}\OperatorTok{=}\DecValTok{10}\NormalTok{, r}\OperatorTok{=}\DecValTok{10}\NormalTok{, l}\OperatorTok{=}\DecValTok{10}\NormalTok{))}
\NormalTok{figure.show() }
\CommentTok{\# HTML(figure.to\_html(include\_plotlyjs=\textquotesingle{}cdn\textquotesingle{}))}
\end{Highlighting}
\end{Shaded}

\begin{verbatim}
Unable to display output for mime type(s): text/html
\end{verbatim}

\begin{verbatim}
Unable to display output for mime type(s): text/html
\end{verbatim}

\hypertarget{d.2.-graph-objects}{%
\subsubsection{(D.2.) Graph Objects}\label{d.2.-graph-objects}}

\begin{Shaded}
\begin{Highlighting}[]
\ImportTok{import}\NormalTok{ plotly.graph\_objects }\ImportTok{as}\NormalTok{ go}

\NormalTok{figure }\OperatorTok{=}\NormalTok{ go.Figure()}
\NormalTok{figure.add\_trace(go.Sunburst(}
\NormalTok{        labels}\OperatorTok{=}\NormalTok{df\_sum[}\StringTok{\textquotesingle{}id\textquotesingle{}}\NormalTok{],}
\NormalTok{        parents}\OperatorTok{=}\NormalTok{df\_sum[}\StringTok{\textquotesingle{}parent\textquotesingle{}}\NormalTok{],}
\NormalTok{        values}\OperatorTok{=}\NormalTok{df\_sum[}\StringTok{\textquotesingle{}value\textquotesingle{}}\NormalTok{],}
\NormalTok{        branchvalues}\OperatorTok{=}\StringTok{\textquotesingle{}total\textquotesingle{}}\NormalTok{,}
\NormalTok{        marker}\OperatorTok{=}\BuiltInTok{dict}\NormalTok{(colorscale}\OperatorTok{=}\StringTok{\textquotesingle{}Rdbu\textquotesingle{}}\NormalTok{),}
\NormalTok{        hovertemplate}\OperatorTok{=}\StringTok{\textquotesingle{}\textless{}b\textgreater{} Country : \%}\SpecialCharTok{\{label\}}\StringTok{ \textless{}/b\textgreater{} \textless{}br\textgreater{} Count : \%}\SpecialCharTok{\{value\}}\StringTok{ \textless{}extra\textgreater{}Population\textless{}/extra\textgreater{}\textquotesingle{}}\NormalTok{,}
\NormalTok{        maxdepth}\OperatorTok{=}\DecValTok{2}\NormalTok{)}
\NormalTok{    )}
\NormalTok{figure.update\_layout(margin}\OperatorTok{=}\BuiltInTok{dict}\NormalTok{(t}\OperatorTok{=}\DecValTok{10}\NormalTok{, b}\OperatorTok{=}\DecValTok{10}\NormalTok{, r}\OperatorTok{=}\DecValTok{10}\NormalTok{, l}\OperatorTok{=}\DecValTok{10}\NormalTok{))}
\NormalTok{figure.show() }
\CommentTok{\# HTML(figure.to\_html(include\_plotlyjs=\textquotesingle{}cdn\textquotesingle{}))}
\end{Highlighting}
\end{Shaded}

\begin{verbatim}
Unable to display output for mime type(s): text/html
\end{verbatim}

\hypertarget{e-communicating-plots-with-json}{%
\subsection{(E) Communicating Plots with
JSON}\label{e-communicating-plots-with-json}}

\begin{center}\rule{0.5\linewidth}{0.5pt}\end{center}

Plotly has in-built function to save figure as json :
\emph{write\_json()}. Following cells show how to write and regenerate
the plots.

\begin{Shaded}
\begin{Highlighting}[]
\NormalTok{figure.write\_json(}\StringTok{"../data/Sunburst\_Chart.json"}\NormalTok{)}
\end{Highlighting}
\end{Shaded}

\begin{Shaded}
\begin{Highlighting}[]
\ImportTok{import}\NormalTok{ json}

\NormalTok{opened\_file }\OperatorTok{=} \BuiltInTok{open}\NormalTok{(}\StringTok{"../data/Sunburst\_Chart.json"}\NormalTok{)}
\NormalTok{opened\_fig }\OperatorTok{=}\NormalTok{ json.load(opened\_file)}

\NormalTok{fig\_ }\OperatorTok{=}\NormalTok{ go.Figure(}
\NormalTok{    data }\OperatorTok{=}\NormalTok{ opened\_fig[}\StringTok{\textquotesingle{}data\textquotesingle{}}\NormalTok{],}
\NormalTok{    layout }\OperatorTok{=}\NormalTok{ opened\_fig[}\StringTok{\textquotesingle{}layout\textquotesingle{}}\NormalTok{]}
\NormalTok{    )}
\NormalTok{fig\_.show()}
\CommentTok{\# HTML(fig\_.to\_html()) }
\end{Highlighting}
\end{Shaded}

\begin{verbatim}
Unable to display output for mime type(s): text/html
\end{verbatim}

\hypertarget{f-custom-plots}{%
\subsection{(F) Custom Plots}\label{f-custom-plots}}

\begin{center}\rule{0.5\linewidth}{0.5pt}\end{center}

In the final section we would see the go.Figure subplots, where fully
customize the plots.

\begin{Shaded}
\begin{Highlighting}[]
\ImportTok{from}\NormalTok{ plotly.subplots }\ImportTok{import}\NormalTok{ make\_subplots}

\NormalTok{fig }\OperatorTok{=}\NormalTok{ make\_subplots(}\DecValTok{1}\NormalTok{, }\DecValTok{2}\NormalTok{, specs}\OperatorTok{=}\NormalTok{[[\{}\StringTok{"type"}\NormalTok{: }\StringTok{"domain"}\NormalTok{\}, \{}\StringTok{"type"}\NormalTok{: }\StringTok{"domain"}\NormalTok{\}]],)}
\NormalTok{fig.add\_trace(go.Sunburst(}
\NormalTok{    labels}\OperatorTok{=}\NormalTok{df\_sum[}\StringTok{\textquotesingle{}id\textquotesingle{}}\NormalTok{],}
\NormalTok{    parents}\OperatorTok{=}\NormalTok{df\_sum[}\StringTok{\textquotesingle{}parent\textquotesingle{}}\NormalTok{],}
\NormalTok{    values}\OperatorTok{=}\NormalTok{df\_sum[}\StringTok{\textquotesingle{}value\textquotesingle{}}\NormalTok{],}
\NormalTok{    branchvalues}\OperatorTok{=}\StringTok{\textquotesingle{}total\textquotesingle{}}\NormalTok{,}
\NormalTok{    marker}\OperatorTok{=}\BuiltInTok{dict}\NormalTok{(colorscale}\OperatorTok{=}\StringTok{\textquotesingle{}sunset\textquotesingle{}}\NormalTok{),}
\NormalTok{    hovertemplate}\OperatorTok{=}\StringTok{\textquotesingle{}\textless{}b\textgreater{} Country : \%}\SpecialCharTok{\{label\}}\StringTok{ \textless{}/b\textgreater{} \textless{}br\textgreater{} Count : \%}\SpecialCharTok{\{value\}}\StringTok{ \textless{}extra\textgreater{}Population\textless{}/extra\textgreater{}\textquotesingle{}}\NormalTok{,}
\NormalTok{    maxdepth}\OperatorTok{=}\DecValTok{2}\NormalTok{), }\DecValTok{1}\NormalTok{, }\DecValTok{1}\NormalTok{)}

\NormalTok{fig.add\_trace(go.Sunburst(}
\NormalTok{    labels}\OperatorTok{=}\NormalTok{df\_count[}\StringTok{\textquotesingle{}id\textquotesingle{}}\NormalTok{],}
\NormalTok{    parents}\OperatorTok{=}\NormalTok{df\_count[}\StringTok{\textquotesingle{}parent\textquotesingle{}}\NormalTok{],}
\NormalTok{    values}\OperatorTok{=}\NormalTok{df\_count[}\StringTok{\textquotesingle{}value\textquotesingle{}}\NormalTok{],}
\NormalTok{    branchvalues}\OperatorTok{=}\StringTok{\textquotesingle{}total\textquotesingle{}}\NormalTok{,}
\NormalTok{    marker}\OperatorTok{=}\BuiltInTok{dict}\NormalTok{(colorscale}\OperatorTok{=}\StringTok{\textquotesingle{}viridis\textquotesingle{}}\NormalTok{),}
\NormalTok{    hovertemplate}\OperatorTok{=}\StringTok{\textquotesingle{}\textless{}b\textgreater{} Country : \%}\SpecialCharTok{\{label\}}\StringTok{ \textless{}/b\textgreater{} \textless{}br\textgreater{} Count : \%}\SpecialCharTok{\{value\}}\StringTok{ \textless{}extra\textgreater{}Cities\textless{}/extra\textgreater{}\textquotesingle{}}\NormalTok{,}
\NormalTok{    maxdepth}\OperatorTok{=}\DecValTok{2}\NormalTok{), }\DecValTok{1}\NormalTok{, }\DecValTok{2}\NormalTok{)}

\NormalTok{fig.update\_layout(margin}\OperatorTok{=}\BuiltInTok{dict}\NormalTok{(t}\OperatorTok{=}\DecValTok{10}\NormalTok{, b}\OperatorTok{=}\DecValTok{10}\NormalTok{, r}\OperatorTok{=}\DecValTok{10}\NormalTok{, l}\OperatorTok{=}\DecValTok{10}\NormalTok{))}
\NormalTok{fig.show()}
\CommentTok{\# HTML(fig.to\_html()) }
\end{Highlighting}
\end{Shaded}

\begin{verbatim}
Unable to display output for mime type(s): text/html
\end{verbatim}

\hypertarget{thank-you}{%
\section{Thank you!}\label{thank-you}}

\begin{itemize}
\item
  References :

  \begin{enumerate}
  \def\labelenumi{\arabic{enumi}.}
  \tightlist
  \item
    \href{https://plotly.com/}{Plotly}
  \item
    \href{https://plotly.com/python/sunburst-charts/}{Sunburst in
    Python}
  \end{enumerate}
\end{itemize}



\end{document}
